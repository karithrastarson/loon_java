The loon project aims to launch thousands of balloons, each carrying expensive equipment and heavy, not to mention dangerous when falling freely. It would be foolish to attempt such a thing before making sure that the end goal is plausible, if not definitely possible. This can be done relatively cheaply using computer simulations, which is exactly the aim of this project.

The balloons are located in the stratosphere so they escape the winds and weather conditions that we experience from the troposphere. The balloons float with stratospheric currents that are more regular than the wind the troposphere and more predictable. The layers are multiple and differ between altitude. This means that the steering of the balloons is done simply by going up or going down. The balloons can decrease or increase their altitude to catch a new wind layer and go to another direction. 

Even though the world has been simplified significantly and many laws of physics overlooked, the simulation should give a good idea as to whether this idea is good or poor. Arguments are provided for all assumptions made during the modelling and their implications.  