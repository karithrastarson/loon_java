This project is like Pandora's box. The idea is so simple and conceivable and creating a model for it sounds like a piece of cake. Circles floating around some area according to some predefined wind speed. But the considerations and details are endless. A long discussion could be had about each detail of the model but that is material for a whole other project. To keep things short and precise some key points have been chosen that are interesting to give a second thought to.
\subsection{Assumptions}
All of the assumptions can be discussed to great lengths and scrutinized, but that is the nature of assumptions.
\subsection{Decentralized vs. Centralized}
There are two ways to implement the control of the balloons. With a centralized algorithm or decentralized, each with its pros and cons. With a centralized solution every balloon must be in contact with the main data source at every given moment. This means that the balloons have to stay close together, and always in connection with a ground station that crunches the numbers and provides the balloons with a decision. The controlling could be more accurate with this method, but also prone to failures, since the balloons rely so heavily on the ground connection.

A decentralized version is on the other hand more forgiving when it comes to connection. Each balloon can survive a temporary blackout and pays attention only to the object within its detection range. The controlling however would be shortsighted and a balloon would not see the forest for the trees in some sense.

The optimal solution is a mixture of centralized and decentralized algorithm. When possible, balloons would be provided with updated status of the entire system; the location of other balloons in the system and updated weather data. This would give the balloons capability to navigate without connection to the ground, but also a frequent overview of the system as a whole. 
\subsection{Access to Data}
\subsection{Further Work}