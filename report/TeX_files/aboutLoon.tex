The Project Loon is carried out by X (formerly Google X), a semi-secret department under the Alphabet umbrella. The project was started unofficially in 2011 and, as with all projects, was small in scope. The first prototypes were constructed with cheap material bought on the Internet and the launches included letting a balloon loose and following it with an antenna, fast car and reckless driving. The project was so secretive that none of the equipment was labeled with Google's name, but rather a sticker that read "If found, return to Paul", and a promise of a reward. With this promise Google hoped to retrieve all crashed balloons during these trial months.

In June 2013 the team announced that it had successfully provided some 50 families with Internet connection through balloons. This announcement served as a proof of concept and the development continued. The lifespan of each balloon was extended with better latex and manufacturing process. The controlling of the balloons has also improved greatly with better analysis of weather data, but in addition to using NOAA's data, the balloons now collect their own weather data.

The project has gone through several stages of improvement and the manufacturing process has been optimized so that a new balloon can be created and launched in a matter of hours. This shows that Google takes this project seriously and is on the brink of scaling significantly. They still have some obstacles to overcome before this project can be launched on a global scale, like air traffic regulations, landing pads and service centers for the balloons and such. But these obstacles pale in comparison to the ones already overcome.