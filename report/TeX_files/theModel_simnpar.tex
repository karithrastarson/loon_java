The model is the structure and relationship between all objects. It is modifiable in some ways to be able to measure and compare performance of different algorithms. The parameters that should be adjusted before simulation are the following:

\begin{description}
\item[WORLD\_SIZE:] The integer dimension of the grid that represents the earth. This number greatly affects the run time of the model as well as accuracy. 
\item[NUMBER\_OF\_BALLOONS:] The maximum number of balloons in the model. The model adds a new balloon automatically during each iteration until this number of balloons is reached. This number will control the range of each balloons as explained in section \ref{theModel_numBal}.
\item[VERTICAL\_SPEED:] The speed of the balloon in directions up and down. This number will be added/subtracted from the balloon's altitude each step, until its motion in that direction is stopped.
\item[NUMBER\_OF\_STEPS:] How many steps the simulation should run. This number should be as big as possible to properly model the behaviour of the system. 
\item[NUMBER\_OF\_CURRENTS:] The number of wind layers, or wind currents, in the system. The more layers there is to choose from the more accurate direction the balloon should be able to choose from.
\item[MIN/MAX\_ALTITUDE:] Numbers used to divide the layers. The total distance between MIN and MAX is divided equally between all the layers in the model. This is used to determine whether a balloon has reached a new wind layer. The units are abstract but should be taken into consideration when choosing vertical speed. The vertical speed corresponds to the units in this MIN/MAX scale.
\item[RANGE: ] The grid of balloons correspond to a certain surface on earth. This parameter dictates the size of the area that one balloon can impact. That is the radius of the area that one balloon can service. 
\item[LIFETIME: ] The maximum lifetime of a balloon is fixed. This is the number of steps each balloon can participate in. This number should be adjusted in accordance with the size of the grid, and the number of steps, since it would be difficult to get any results in a model where the balloons are removed just as they're about to get in place. 
\item[COMMUNICATION\_RADIUS: ] The communication radius of the model is another parameter applied in the second two algorithms, but that controls how far the balloons can communicate between each other to share information about location and wind conditions.
\end{description}
When all objects have been created and configured, the simulation is run. 
The simulation moves all balloons according to the established stratosphere. A simulation starts with no balloons in the system and repeats the following steps:
\begin{enumerate}
    \item Create balloon? Is the system saturated with balloons. The parameter \textit{NUMBER\_OF\_BALLOONS} is used to determine whether a new balloon should be spawned or not.
    \item Apply decisions: All balloons decide, using a control algorithm, whether they should start moving up or down or stay in the current altitude. 
    \item Apply currents: When the decision has been made, all the balloons are moved according to the wind layer they are currently located in. Furthermore they are moved up or down according to the model parameter for vertical speed. 
    \item Update statistics: The statistics of the simulation are gathered on the fly so after each iteration the data is re-evaluated and stored for further analysis.
\end{enumerate}

