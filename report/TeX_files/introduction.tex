Nowadays the Internet is no longer only a source of entertainment and leisurely activities, but an essential part of everyday operations. There should be no need to list all of the important roles this technology plays in people's everyday routine, but this list is actually getting longer everyday. In light of these facts it is astounding to read the annual report from the \textit{Internet Society} which states that in 2013 60\% percentage of the global population was still not connected to the Internet. 

The race has begun and many technology giants are now pursing to hook the rest of the world up with an Internet connection for a reasonable. But how can this be achieved? The companies each have their way of implementing this but they have one thing in common: They have all turned the usual way of communicating with Internet provider upside down. As a person moved around he would connect with different radio towers, based on his current location. With the balloons this example is overturned. As the person stands still, he communicated with different balloons as the hover above and change location.

Whether the provider of the signal is a balloon, drone or some other device, the principal is the same. The technology has been strapped to a mobile object which moves around in the stratosphere and provides rural and remote ares with a 4G Internet connection. The aim of this project is to model the efforts being made by Google, but their team is focusing on balloons. The model will generalize the behaviour and simplify the universe to some extent. Reasons for all assumptions will be provided as they are made.