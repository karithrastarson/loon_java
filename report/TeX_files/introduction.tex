Nowadays the Internet is no longer only a source of entertainment and leisurely activities, but an essential part of everyday operations. There should be no need to list all of the important roles this technology plays in people's everyday routine, but this list is actually getting longer every day. In light of these facts it is astounding to read the annual report from the \textit{Internet Society} which states that in 2013 60\% \citep{Brown} of the global population was still not able to connect to the Internet. 

The race has begun and many technology giants are now pursuing to hook the rest of the world up with an Internet connection for a reasonable price. But how can this be achieved? The companies each have their way of implementing this but they are all headed in the same general direction; up. They have all turned the usual way of communicating with an Internet provider upside down. As a person moved around he would connect with different radio towers, based on his current location. With the balloons this example is turned on its head. As the person stands still, he communicates with different balloons as they hover above and change location.

Whether the provider of the signal is a balloon, drone or some other device, the principal is the same. The technology has been strapped to a mobile object which moves around in the stratosphere and provides rural and remote ares with a 4G Internet connection. The aim of this project is to establish a foundation for a model that simulates the efforts being made by Google, but their team is focusing on balloons. The model will generalize the behaviour and simplify the universe to some extent. Reasoning for all assumptions will be provided as they are made. Rather than honing in on a perfect solution, several control algorithms are developed for the balloons  during this project and they aim to explore different aspects of the problem in question. Each algorithm is explained and tested, and thoughts given on its pitfalls and how they could be improved.

The data used is not actual weather data but randomly, but realistically, generated mock data. To further enhance the accuracy of the model actual stratospheric weather data could be used as an input.